\documentclass[a4paper]{article}
\usepackage[utf8]{inputenc}
\usepackage{lmodern}
\usepackage[english]{babel}
\usepackage[T1]{fontenc}
\usepackage{amsmath,amsthm,amssymb}
\usepackage[colorlinks=true,allcolors=black]{hyperref}
\usepackage{csquotes}
\usepackage{enumitem}

\usepackage{tikz}
\usepackage{pgfplots}
\pgfplotsset{compat=1.18}
\usetikzlibrary{datavisualization}
\usetikzlibrary{datavisualization.formats.functions}
\usetikzlibrary{matrix, calc, decorations}


% link to database file
\usepackage[backend=biber]{biblatex}
\addbibresource{/home/ralf/Projekte/LitDatabase/liblatexbase.bib}

\usepackage[left=1in,right=1in,top=1in,bottom=.5in,includeheadfoot]{geometry}
%\usepackage[colorlinks=true,allcolors=black]{hyperref}
\usepackage{url}

\setlength{\parindent}{0pt}
\setlength{\parskip}{2ex}

\renewcommand{\arraystretch}{1.2}  %vertical stretch of arrays

\newcommand{\BF}{\mathbb{F}}

% for conversion see https://www.glukhov.org/de/post/2025/11/converting-latex-to-markdown-tools-and-workflows/
% pandoc document.tex -t markdown+tex_math_dollars -o document.md
% pandoc document.tex --bibliography=refs.bib --citeproc -o document.md
% pandoc gf2vs.tex -t markdown_strict+tex_math_dollars+raw_tex --bibliography=/home/ralf/Projekte/LitDatabase/liblatexbase.bib --citeproc -o gf2vs.md

\title{Vector space $\BF_2^n$}
\author{Ralf Pöppel \\ \url{mailto:ralf@poeppel-familie.de}}
\date{2026-01-12}
\begin{document}
\maketitle

\begin{abstract}
This article is a supplemental documentation to the package gf2vs. 
It descibes the vector space $\BF_2^n$ based on the finite field of order 2 or Galois field $GF(2)$ of size $n$. \cite{wikipedia_en_2026_GaloisField}
\end{abstract}

\section*{Field $\BF_2$}
The finite field of order 2 has 2 elements $\BF_2 = \{0, 1\}$ and the operations addition  $+$ and multiplication $\cdot$. For the
definition see equation (\ref{def:fieldoperations}).
\begin{equation}
\begin{array}{r l l l l}
+ : & 0 + 0 = 0, & 0 + 1 = 1, & 1 + 0 = 1, & 1 + 1 = 0, \\
\cdot : & 0 \cdot 0 = 0, & 0 \cdot 1 = 0, & 1 \cdot 0 = 0, & 1 \cdot 1 = 1. \\
\end{array} \label{def:fieldoperations}
\end{equation}
Each of the 2 operations of the field $\BF_2$ satisfy the group axioms \cite{wikipedia_en_2026_GroupMathematics} for the groups $G_+ : (\BF_2, +)$ and $G_{\cdot} : (\BF_2, \cdot)$. For reference the group axioms are repeated here. We use the symbol $\circ$ to denote the binary operations $+, \cdot$.
\begin{description}
\item[Associativity]  \ \\
$\forall a, b, c \in G: (a \circ b) \circ c = a \circ (b \circ c)$.
\item[Identity element $e$] \ \\
$\exists e \in G, \forall a \in G:  e \circ a = a \text{ and }  a \circ e = a$, $e$ is unique.
\item[Inverse element $a^{-1}$] \ \\
$\forall a \in G \enspace \exists b \in G : a \circ b = e \text{ and } b \circ a = e, e$ identity element, $ b $ is unique $\forall a$, notation $b = a^{-1}$.
\end{description}

We can look at the field from an algebraic point of view or from a logic view. In logic the field can be seen as the boolean variables $F = 0$ and $T = 1$. The boolean operations are disjunction $\vee$ \cite{wikipedia_en_2025_InclusiveOr}, 
contravalence~$\oplus$~ \cite{wikipedia_en_2025_ExclusiveOr} and 
conjunction $\wedge$ \cite{wikipedia_en_2025_LogicalConjunction} the definition is repeated in equation (\ref{def:booleanoperation}).
\begin{equation}
\begin{array}{r l l l l}
\vee : & 0 \vee 0 = 0, & 0 \vee 1 = 1, & 1 \vee 0 = 1, & 1 \vee 1 = 1, \\
\oplus : & 0 \oplus 0 = 0, & 0 \oplus 1 = 1, & 1 \oplus 0 = 1, & 1 \oplus 1 = 0, \\
\wedge : & 0 \wedge 0 = 0, & 0 \wedge 1 = 0, & 1 \wedge 0 = 0, & 1 \wedge 1 = 1. \\
\end{array} \label{def:booleanoperation}
\end{equation}
Please note the operations $\oplus$ and $\wedge$ are identically defined as $+$ and $\cdot$ and hence satisfy the group axioms.
But the operation $\vee$ does not satisfy the group axioms, there is no inverse element. In the remaining chapters we will use the 
notation $+, \cdot$ for the operations only.

\section*{Vector Space $\BF_2^n$}
We define the vector space 
$\BF_2^n$ over the field $\BF_2$ as set $V$ of vectors $v$ of $n$ elements of the field together with the 
binary operation addition~$u + v = w, \enspace u, v, w \in V$ and the binary function scalar multiplication~$a \cdot v = w, \enspace a \in \BF_2, v, w \in V$. We apply the addition element-wise and we multiply the scalar with each element of the vector.
This definition is similar to the definition in \cite{wikipedia_en_2025_VectorSpace}.



\newpage
\printbibliography

\end{document}

