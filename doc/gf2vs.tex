\documentclass[a4paper]{article}
\usepackage[utf8]{inputenc}
\usepackage[english]{babel}
\usepackage[T1]{fontenc}
\usepackage{amsmath,amsthm,amssymb}
\usepackage{csquotes}
\usepackage{enumitem}
\usepackage{newtxtext,newtxmath}

% link to database file
\usepackage[backend=biber]{biblatex}
\addbibresource{/home/ralf/Projekte/LitDatabase/liblatexbase.bib}

\usepackage[left=1in,right=1in,top=1in,bottom=.5in,includeheadfoot]{geometry}

\usepackage[colorlinks=true,allcolors=black]{hyperref}
\setcounter{biburllcpenalty}{9000}% Kleinbuchstaben
\setcounter{biburlucpenalty}{9000}% Großbuchstaben

\setlength{\parindent}{0pt}
\setlength{\parskip}{2ex}

\renewcommand{\arraystretch}{1.2}  %vertical stretch of arrays

\newcommand{\BE}{\mathbb{E}}
\newcommand{\BF}{\mathbb{F}}
\newcommand{\BZ}{\mathbb{Z}}
\newcommand{\0}{\vmathbb{0}}
\newcommand{\1}{\vmathbb{1}}

% for conversion see https://www.glukhov.org/de/post/2025/11/converting-latex-to-markdown-tools-and-workflows/
% pandoc gf2vs.tex -t markdown_strict+tex_math_dollars+raw_tex --bibliography=/home/ralf/Projekte/LitDatabase/liblatexbase.bib --citeproc -o gf2vs.md

\title{Vector space $\BF_2^n$}
\author{Ralf Poeppel \\ \url{mailto:ralf@poeppel-familie.de}}
\date{2026-01-14}
\begin{document}
\maketitle

\begin{abstract}
This article is a supplemental documentation to the go package gf2vs \cite{rp:gopkg:gf2vs}. The package implements data types and functions 
modeling the vector space $\BF_2^n$.
The vector space $\BF_2^n$  of size $n$ is based on the finite field of order 2 or Galois field $GF(2)$ \cite{wikipedia_en_2026_GaloisField}. We use $GF(2)$ to model binary values, or bits. We consider the properties of the vector space of bit vectors.
\end{abstract}

\section*{Field $\BF_2$}
The finite field of order 2 has 2 elements $\BF_2 = \{0, 1\}$ and the operations addition~ $+$ and multiplication~$\cdot$. For the
definition see equation (\ref{def:fieldoperations}).
\begin{equation}
\begin{array}{r l l l l}
+ : & 0 + 0 = 0, & 0 + 1 = 1, & 1 + 0 = 1, & 1 + 1 = 0, \\
\cdot : & 0 \cdot 0 = 0, & 0 \cdot 1 = 0, & 1 \cdot 0 = 0, & 1 \cdot 1 = 1. \\
\end{array} \label{def:fieldoperations}
\end{equation}
We may use the notation $a b$ instead of $a \cdot b$ omitting the multiplication sign if there is no ambiguity.

Each of the 2 operations of the field $\BF_2$ satisfy the group axioms \cite{wikipedia_en_2026_GroupMathematics} for the groups $G_+ : (\BF_2, +)$ and $G_{\cdot} : (\BF_2, \cdot)$. in addition both operations are commutative. For reference the group axioms are repeated here. We use the symbol $\circ$ to denote the binary operations $+, \cdot$.
\begin{description}
\item[Associativity]  \ \\
$\forall a, b, c \in G: (a \circ b) \circ c = a \circ (b \circ c)$.
\item[Identity element $e$] \ \\
$\exists e \in G, \forall a \in G:  e \circ a = a \text{ and }  a \circ e = a$, $e$ is unique.
\item[Inverse element $a^{-1}$] \ \\
$\forall a \in G \enspace \exists b \in G : a \circ b = e \text{ and } b \circ a = e, e$ identity element, $ b $ is unique $\forall a$, notation $b = a^{-1}$.
\item[Commutivity] \ \\
$a \circ b= b \circ a$.
\end{description}

We can look at the field from an algebraic point of view or from a logic view. In logic the field can be seen as the boolean variables $F = 0$ and $T = 1$. The boolean operations are disjunction $\vee$ \cite{wikipedia_en_2025_InclusiveOr}, 
contravalence~$\oplus$~ \cite{wikipedia_en_2025_ExclusiveOr} and 
conjunction $\wedge$ \cite{wikipedia_en_2025_LogicalConjunction}. The definition is repeated in equation (\ref{def:booleanoperation}).
\begin{equation}
\begin{array}{r l l l l}
\vee : & 0 \vee 0 = 0, & 0 \vee 1 = 1, & 1 \vee 0 = 1, & 1 \vee 1 = 1, \\
\oplus : & 0 \oplus 0 = 0, & 0 \oplus 1 = 1, & 1 \oplus 0 = 1, & 1 \oplus 1 = 0, \\
\wedge : & 0 \wedge 0 = 0, & 0 \wedge 1 = 0, & 1 \wedge 0 = 0, & 1 \wedge 1 = 1. \\
\end{array} \label{def:booleanoperation}
\end{equation}
Please note the operations $\oplus$ and $\wedge$ are identically defined as $+$ and $\cdot$ and hence satisfy the group axioms.
But the operation $\vee$ does not satisfy the group axioms, there is no inverse element. In the remaining chapters we will use the 
notation $+, \cdot, \vee$ for the operations only.

\section*{Vector Space $\BF_2^n$}
We define the vector space 
$\BF_2^n$ over the field $\BF_2$ as set $V$ of vectors $v$ of $n$ elements of the field together with the 
binary operation addition and the binary function scalar multiplication (\ref{def:vectoroperations}).
\begin{equation}
u + v = w, \enspace u, v, w \in V,  \quad
a \cdot v = w, \enspace a \in \BF_2, v, w \in V. 
\label{def:vectoroperations}
\end{equation}
 We apply the addition element-wise and we multiply the scalar with each element of the vector.
This definition is similar to the definition in \cite{wikipedia_en_2025_VectorSpace}.

We use the notation $(v_i) := v$ for the vector $v$ with the components $v_i$ .

In addition we define 2 constants of $\BF_2^n$:
\begin{description}
\item[Zeros]  \ \\
$\0$ Zero vector were all components are $0$.
\item[Ones] \ \\
$\1$ Vector were all components are $1$.
\end{description}

The axioms of a vector space are satisfied  \cite{wikipedia_en_2025_VectorSpace}:
\begin{description}
\item[Associativity of vector addition]  \ \\
$u + (v + w) = (u + v) + w, \enspace \forall u, v, w \in \BF_2^n$.
\item[Commutativity of vector addition] \ \\
$u + v = v + u, \enspace \forall u, v \in \BF_2^n$.
\item[Identity element of vector addition] \ \\
$\exists \0 \in \BF_2^n : v + \0 = v, \enspace \forall v \in \BF_1^n$.
\item[Inverse elements of vector addition] \ \\
$\forall v \in \BF_2^n \enspace \exists -v \in \BF_2^n : v + (-v) = \0, -v = v$, \enspace each vector is its own additive inverse.
\item[Compatibility of scalar multiplication with field multiplication] \ \\
$a (b v) = (a b ) v, \enspace a, b \in \BF_2, v \in \BF_2^n$.
\item[Identity element of scalar multiplication] \ \\
$1 v = v, \enspace 1 \in \BF_2, v \in \BF_2^n$, \enspace $1$ is the multiplicative identity of $\BF_2$.
\item[Distributivity of scalar multiplication with respect to vector addition] \ \\
$a (u + v) = a u + a v, \enspace a \in \BF_2, u, v \in \BF_2^n$.
\item[Distributivity of scalar multiplication with respect to field addition] \ \\
$(a + b) v = a v + b v, \enspace a, b \in \BF_2, v \in \BF_2^n$.
\end{description}

In this vector space we are not limited to the operations vector addition and scalar multiplication. We can use the boolean operations too.
\begin{description}
\item[Complement, Not]  \ \\
$\overline v = \1 - v = \1 + v$, swap all bits.
\item[Disjunction, Or] \ \\
$u \vee v = (u_i) \vee (v_i) = (u_i \vee v_i)$, element wise Or.
\item[Contravalence, Xor] \ \\
$u \oplus v = u + v = (u_i) + (v_i) = (u_i + v_i)$, element wise xor, duplicate of vector addition.
\item[Conjunction, And] \ \\
$u \wedge v = (u_i) \cdot (v_i) = (u_i \cdot v_i)$, element wise And.
\end{description}
As we apply the operations element wise, we satisfy the laws of associativity and commutativity.

We use some more definitions to cover the further properties of a vector space:
\begin{description}
\item[Unit vector]  \ \\
We define the unit vectors $e_i, i = 1, \dots, n$ of the vector space as the vectors where the $i$th~element is $x_i = 1$ and all other elements are $0$. 
\begin{flalign*}
& e_i = (x_k), & \\
& x_k = \left\{ \begin{array}{ll} 1, & k = i, \\ 0, & k \ne i,  \\ \end{array} \right. x_k \in \BF_2, \enspace e_i \in \BF_2^n. & \\
\end{flalign*}
\item[Generating system] \ \\
We define the subspace $\BE = \{e_i\}$, $\BE  \subset \BF_2^n$ of vectors $e_i$.
The subspace $\BE$ forms a generating system. Obviously each vector $v$ of $\BF_2^n$ is a linear combination of the scalars $a_i,$ and the $e_i$.
\begin{flalign*}
&  v = \sum_{i=1}^n a_i e_i, \enspace a_i \in \BF_2, \enspace e_i \in \BE, \quad \forall v \in \BF_2^n. &
\end{flalign*}
So the subset $\BE$ is a span of $\BF_2^n$. In this vector space it is the only span. And the decomposition of a vector $v$ in a linear combination
of unit vectors $e_i$ is unique.
\item[Basis] \ \\
The subspace $\BE$ is the one and only basis of the vector space $\BF_2^n$. 
\item[Index] \ \\
We name $i = 1, \dots n$ of $e_i$ the index of a unit vector in the basis.
\item[Norm] \ \\
We define the Norm $|v|$ of a vector $v \in \BF_2^n$ to be its Hamming weight \cite{wikipedia_en_2025_HammingWeight}. In this case the count of ones of the vector. The value of the norm is an element of the set $\{0, 1, \dots n\} \ne \BF_2, n > 2$, In contrast to usual vector spaces for example on $\BZ$, where the norm of a vector is an element of~$\BZ$.
\item[Inner product]
We define the inner product of 2 vectors, to be the norm of the product of 2 vectors: \\
$< u, v > = | u \cdot v |$.
\item[Orthogonality] \ \\
$< u, v > = 0$, \\
we say 2 vectors are orthogonal if the inner product is $0$.  Please note the inner product of any vector with $\0$ is $0$.
\end{description}

\newpage
\printbibliography

\end{document}

