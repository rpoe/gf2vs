\documentclass[a4paper]{article}
\usepackage[utf8]{inputenc}
\usepackage[english]{babel}
\usepackage[T1]{fontenc}
\usepackage{amsmath,amsthm,amssymb}
\usepackage{csquotes}
\usepackage{enumitem}
\usepackage{newtxtext,newtxmath}

% link to database file
% \usepackage{natbib} not working
\usepackage[backend=biber]{biblatex}
\addbibresource{/home/ralf/Projekte/LitDatabase/liblatexbase.bib}

\usepackage[left=1in,right=1in,top=1in,bottom=.5in,includeheadfoot]{geometry}
\usepackage[colorlinks=true,allcolors=black]{hyperref}

\setcounter{biburllcpenalty}{9000}% Kleinbuchstaben
\setcounter{biburlucpenalty}{9000}% Großbuchstaben


\setlength{\parindent}{0pt}
\setlength{\parskip}{2ex}

\renewcommand{\arraystretch}{1.2} % vertical stretch of arrays

\newcommand{\BC}{\mathbb{C}}
\newcommand{\BE}{\mathbb{E}}
\newcommand{\BF}{\mathbb{F}}
\newcommand{\BQ}{\mathbb{Q}}
\newcommand{\BR}{\mathbb{R}}
\newcommand{\BZ}{\mathbb{Z}}
\newcommand{\0}{\vmathbb{0}}
\newcommand{\1}{\vmathbb{1}}

% for conversion see https://www.glukhov.org/de/post/2025/11/converting-latex-to-markdown-tools-and-workflows/
% pandoc gf2vs.tex -t markdown_strict+tex_math_dollars+raw_tex --bibliography=/home/ralf/Projekte/LitDatabase/liblatexbase.bib --citeproc -o gf2vs.md

\title{Vector space of bit vectors}
\author{Ralf Poeppel \\ \url{mailto:ralf@poeppel-familie.de}}
\date{2026-02-04}

\begin{document}

\maketitle

\begin{abstract}
This article is a supplementary documentation to the Go package~\texttt{gf2vs}~\cite{rp:gopkg:gf2vs}. 
The package implements data types and functions for the vector space of bit vectors.
\end{abstract}

\section{Introduction}

Bit vectors are very common in computer science. They are used in for integers, combinatorial algorithms, coding theory
and for logical and arithmetic operations \cite{knuth:taocp_v4_f1, Hill2004}. 
All aspects of the vector space of bit vectors are examined relatively rarely.

Bits are based on the the finite set of integers of order $2$. This set $\{0, 1\}$ with the operations addition and multiplication modulo 2 satisfy the axioms of a field. This finite field is named Galois Field \cite{Lidl2000, wikipedia_en_2026_GaloisField}\footnote{We cite Wikipedia for reused wordings.} $GF(2) = \BF_2$. On this field there is the vector space $\BF_2^n$. 

The aim of this article is to document the properties of the vector space of bit vectors $\BF_2^n$, as implemented in the Go package~
\texttt{gf2vs}. This is a brief reference of the properties collected from several sources.

\section{\texorpdfstring{Field $GF(2)$}{Field GF(2)}}

\subsection{Supporting Set}

The supporting set of $GF(2) = \BF_2$ is 
\begin{equation}
\BZ_2 = \BZ/\BZ2 = \{0, 1\} \subset \BZ \subset \BR
\end{equation}
the subset $\BZ_2$ of $\BZ$, which is a subset of  $\BR$. This set is equal to $\BZ/\BZ2$ the cyclic set of order 2. This set hold the values of a bit in computer science. 
In logic we have the boolean values False $F = 0$ and True~ $T = 1$~\cite{knuth:taocp_v4_f0}. 


\subsection{Operations}
The operations of $\BF_2 = \BZ/\BZ2$ are defined modulo~2 see \cite{Fischer2020} ch. 2.2.6 and \cite{Hill2004}.

The operations addition and multiplication of the field $\BF_2$ satisfies the group 
axioms~\cite{Fischer2020, Lidl2000, wikipedia_en_2026_GroupMathematics}, both operations are commutative.

Similarly the operations addition and multiplication of the field $\BR$ satisfies the group 
axioms~\cite{Fischer2020}, both operations are commutative.

Please note the different definition of the addition in $\BF_2$ and $\BR$.

For reference we give here only the operations for $\BF_2$.

\subsubsection{Negation}

\begin{equation}
\begin{array}{r l l l l}
- : \BF_2 \rightarrow \BF_2, & - x = x, & -0 = 0, & -1 = 1. \\
\end{array}
\end{equation}
The negation of $1$ in $\BF_2$ is computed as $(- 1)\mod 2 = 1$

\subsubsection{Complement}

\begin{equation}
\begin{array}{r l l l l}
\neg : \BF_2 \rightarrow \BF_2, & \neg x = 1 - x, & \neg 0 = 1, & \neg1 = 0. \\
\end{array}
\end{equation}

\subsubsection{Absolut-value}
We define the mapping absolute-value $| x |$ of  an element $x$ of $\BF_2$ to $\BR$:
\begin{equation}
\begin{array}{r l l l l}
| x | : \BF_2 \rightarrow \BR, &| 0 | = 0, & | 1  | = 1. \\
\label{def:value}
\end{array}
\end{equation}
We use this mapping, when we need the default classical definition of the addition as in $\BR$.

\subsubsection{Addition}
The addition is named exclusive disjunction in logic and XOR \cite{knuth:taocp_v4_f0, wikipedia_en_2025_ExclusiveOr}
in computer science.
The definition of addition is given in equation \ref{def:opaddition} obeying $(1+1) \mod 2 = 0$.
\begin{equation}
\begin{array}{r l l l l}
+ : \BF_2 \times \BF_2 \rightarrow \BF_2, & 0 + 0 = 0, & 0 + 1 = 1, & 1 + 0 = 1, & 1 + 1 = 0, \\
\end{array}
\label{def:opaddition}
\end{equation}

The group axioms \cite{Fischer2020} ch. 2.2.8 for the Group $G = \BF_2$ and the operation addition are satisfied:
\begin{description}
\item[Associativity] \ \\
$\forall a, b, c \in G: (a + b) + c = a + (b + c)$. 
\item[Identity element $e = 0$] \ \\
$\exists e \in G, \forall a \in G: e + a = a \text{ and } a + e = a$, $e = 0$, $e$ is unique.
\item[Inverse element $(-a) = a$] \ \\
$\forall a \in G \enspace \exists (-a) \in G : a + (-a) = e \text{ and } (-a) + a = e$, $e$ identity element, $(-a) = a$ is unique for each $a$.
\item[Commutativity] \ \\
$a + b = b + a$.
\end{description}
So $\BF_2$ with the operation addition is an abelian group.

\subsubsection{Multiplication}
The multiplication is named conjunction in logic and 
AND \cite{knuth:taocp_v4_f0, wikipedia_en_2025_LogicalConjunction} in computer science.
The multiplication is identically defined as in $\BZ$.
\begin{equation}
\begin{array}{r l l l l}
\cdot : \BF_2 \times \BF_2 \rightarrow \BF_2, & 0 \cdot 0 = 0, & 0 \cdot 1 = 0, & 1 \cdot 0 = 0, & 1 \cdot 1 = 1. \\
\end{array}
\label{def:opmultiplication}
\end{equation}

We may use the notation $ab$ instead of $a \cdot b$, omitting the multiplication sign if there is no ambiguity.

The group axioms for the Group $G = \BF_2$ and the operation multiplication are satisfied:
\begin{description}
\item[Associativity] \ \\
$\forall a, b, c \in G: (a \cdot b) \cdot c = a \cdot (b \cdot c)$.
\item[Identity element $e = 1$] \ \\
$\exists e \in G, \forall a \in G: e \cdot a = a \text{ and } a \cdot e = a$, $e= 1$, $e$ is unique.
\item[Inverse element $a^{-1} = a$] \ \\
$\forall a \in G, a \ne 0, \enspace \exists a^{-1} \in G : a \cdot a^{-1} = e \text{ and } a^{-1}b \cdot a = e$, $e$ identity element, $a^{-1} = 1$ is the only inverse element.
\item[Commutativity] \ \\
$a \cdot b = b \cdot a$.
\end{description}
So $\BF_2$ with the operation multiplication is an abelian group.


\subsubsection{Disjunction}

In boolean logic we have the operation disjunction, named OR in computer science \cite{knuth:taocp_v4_f0, wikipedia_en_2025_InclusiveOr}.

\begin{equation}
\begin{array}{r l l l l}
\vee : \BF_2 \times \BF_2 \rightarrow \BF_2, & 0 \vee 0 = 0, & 0 \vee 1 = 1, & 1 \vee 0 = 1, & 1 \vee 1 = 1 \\
\end{array}
\label{def:booleanoperation}
\end{equation}

The operation $\vee$ does not satisfy the group axioms; there is no inverse element. 

\subsection{Field axioms}

The set $K := \BF_2$ with the operations addition and multiplication satisfies the field axioms \cite{Fischer2020, Hill2004}. We use~$K$ as symbol for any field satisfying the field axioms.
\begin{description}
\item[K1] $K$ with the addition $+$ is an abelian group.
\item[K2] $K^* := K\setminus \{0\}$ with the multiplication $\cdot$ for every element of $K^*$ is an abelian group.
\item[K3] distributive property \cite{Axler2024, wikipedia_en_2025_DistributiveProperty} is satisfied $\forall a, b, c \in K$
\begin{equation}
\begin{array}{c}
a \cdot (b + c) = a \cdot b + a \cdot c, \\
(a + b) \cdot c = a \cdot c + b \cdot c. \\
\end{array}
\end{equation}
\end{description}

\section{\texorpdfstring{Vector space $\BF_2^n$}{Vector space}}

\subsection{Vectors}
Bit vectors are the elements of the vector space.
We define a bit vector $x$ of size $n$ as a tupel $(x_i)$ of values $x_i \in \BF_2$:
\begin{equation}
x := (x_i) := (x_1, x_2, \dots, x_n), \quad \forall x_i \in \BF_2
\end{equation}

We define the set of all bit vectors of size $n$ see \cite{Fischer2020} 2.4.1:
\begin{equation}
\BF_2^n := \{x =  (x_1, \dots, x_n) : x_i \in \BF_2\}
\end{equation}

In addition we define two distinguished constant elements of $\BF_2^n$:
\begin{description}
\item[Zero] $\0$ zero vector, all components are $0$.
\item[Ones] $\1$ ones vector, all components are $1$.
\end{description}

\subsection{Operations on vectors}

We define bitwise operations on the bit vectors $x, y, z$ see \cite{Fischer2020} 2.4.1 and \cite{knuth:taocp_v4_f1} (1), (2), (3).
\begin{equation}
\left.
\begin{array}{rrcrl}
- : \BF_2^n \rightarrow \BF_2^n, & - x = x & \Leftrightarrow & - x_i = x_i \\
\sim : \BF_2^n \rightarrow \BF_2^n, & \sim x = y & \Leftrightarrow & \neg x_i = y_i \\
|\ |: \BF_2^n \rightarrow \BR^n, & |x| = y & \Leftrightarrow & x_i = y_i, \\
\oplus : \BF_2^n \times \BF_2^n \rightarrow \BF_2^n,  & x \oplus y = z & \Leftrightarrow & x_i + y_i = z_i, \\
\& : \BF_2^n \times \BF_2^n \rightarrow \BF_2^n,  & x \& y = z  & \Leftrightarrow & x_i \cdot y_i = z_i, \\
| : \BF_2^n \times \BF_2^n \rightarrow \BF_2^n,  & x | y = z  & \Leftrightarrow & x_i \vee y_i = z_i,   \\
\cdot : \BF_2 \times \BF_2^n \rightarrow \BF_2^n, & \lambda \cdot x = y  & \Leftrightarrow & \lambda \cdot x_i  = y_i,   \\
\end{array} 
\right\} i = 1, \dots, n.
\label{def:vectoroperations}
\end{equation}
We define the operation $| \ |$ for formal mapping of a bit vector from $\BF_2^n$ to $\BR^n$.

For the constants we have: $ \sim \0 = \1$ and $\sim \1 = \0$.

We adopt the main identities from \cite{knuth:taocp_v4_f1} (4), \dots, (14) for bit vectors of size $n$ here:
\begin{flalign}
& x \oplus y =  y \oplus x, \quad x \& y = y \& x, \quad x | y = y | x;  &\\
& (x \oplus y) \oplus z = x \oplus (y \oplus z), \quad (x \& y) \& z = x \& (y \& z), \quad (x | y) | z = x | (y | z); &\\
& (x \oplus y) \& z = (x \& z) \oplus (y \& z); &\\
& (x \& y) | z = (x | z) \& (y | z), \quad (x | y) \& z = (x | z) \& (y | z); &\\
& x \oplus y = (x \& y) \oplus (x | y); &\\
& (x \& y) | x = x, \quad (x | y) \& x = x; \\
& x \oplus \0 = x, \quad x \& \0 = \0, \quad x | \0 = x; & \\
& x \oplus x = \0, \quad x \& x = x, \quad x | x = x; & \\
& x \oplus \1 = \sim x, \quad x \& \1 = x, \quad x | \1 = \1; & \\
& x \oplus (\sim x) = \1, \quad x \& (\sim x) = \0, \quad x | (\sim x) = \1; & \\
& - (x \oplus y) = (\sim x) \oplus y = x \oplus (\sim y), \quad \sim ( x \& y) = (\sim x) | (\sim y), \quad \sim(x | y) = (\sim x) \& (\sim y); &
\end{flalign}

\subsection{Axioms of vector space}
The set $\BF_2^n$ with the binary operation of vector addition $\oplus$ and the binary function of scalar multiplication $\cdot$, as given in~\eqref{def:vectoroperations}, defines a vector space see \cite{Fischer2020, wikipedia_en_2025_VectorSpace}. 

The axioms of a vector space are satisfied for $\BF_2^n$:
\begin{description}
\item[Associativity of vector addition] \ \\
$u \oplus (v \oplus w) = (u \oplus v) \oplus w, \enspace \forall u, v, w \in \BF_2^n$.
\item[Commutativity of vector addition] \ \\
$u \oplus v = v \oplus u, \enspace \forall u, v \in \BF_2^n$.
\item[Identity element of vector addition] \ \\
$\exists \0 \in \BF_2^n : v \oplus \0 = v, \enspace \forall v \in \BF_2^n$.
\item[Inverse elements of vector addition] \ \\
$\forall v \in \BF_2^n \enspace \exists -v \in \BF_2^n : v \oplus (-v) = \0$, and $-v = v$, i.e.\ each vector is its own additive inverse.
\item[Compatibility of scalar multiplication with field multiplication] \ \\
$\lambda (\eta v) = (\lambda \eta) v, \enspace \lambda, \eta \in \BF_2, \; v \in \BF_2^n$.
\item[Identity element of scalar multiplication] \ \\
$1 v = v, \enspace 1 \in \BF_2, \; v \in \BF_2^n$, where $1$ is the multiplicative identity of $\BF_2$.
\item[Distributivity of scalar multiplication with respect to vector addition] \ \\
$\lambda (u \oplus v) = \lambda u \oplus \lambda v, \enspace \lambda \in \BF_2, \; u, v \in \BF_2^n$.
\item[Distributivity of scalar multiplication with respect to field addition] \ \\
$(\lambda + \eta) v = \lambda v + \eta v, \enspace \lambda, \eta \in \BF_2, \; v \in \BF_2^n$.
\end{description}

In the vector space $\BF_2^n$ we are not limited to the operations vector addition and scalar multiplication. We can use the Boolean operations as well.

\begin{description}
\item[Negation, Complement, Not] \ \\
$\sim v = \1 - v = \1 \oplus v$, swap all bits.
\item[Disjunction, Or] \ \\
$u | v = (u_i) | (v_i) = (u_i \vee v_i)$, element-wise Or.
\item[Exclusive or, Xor] \ \\
$u \oplus v = u \oplus v = (u_i) \oplus (v_i) = (u_i + v_i)$, element-wise Xor, equal to vector addition.
\item[Conjunction, And] \ \\
$u \wedge v = (u_i) \cdot (v_i) = (u_i \cdot v_i)$, element-wise And.
\end{description}

As we apply the operations element-wise $i = 1, \dots n$,
we satisfy the laws of associativity and commutativity inherited from the field.

\subsection{Vector space base}
We give here the definition of the base, the norm and the scalar product implemented in the Go package.
The symbol $K$ is used in definitions applicable by each of the fields $\BF_2^n, \BQ, \BR, \BC$.

\subsubsection{Unit vector}
We define the unit vectors $e_i$, $i = 1, \dots, n$, of the vector space as the vectors
where the $i$th element is $x_i = 1$ and all other elements are $0$.
\begin{equation}
e_i = (x_k),
\enspace e_i \in K^n,
\enspace x_k \in K,
\enspace x_k = \left\{
\begin{array}{lll}
1, & k = i, & \text{identity element of multiplication},  \\
0, & k \ne i, & \text{identity element of addition}. \\
\end{array}
\right.
\end{equation}
Please note the $e_i$ are linearly independent.

We observe the identity elements are identical for the fields and the unit vectors are
identical for all vector spaces over a field $K$.

\subsubsection{Generating system}
We define the subset $\BE := \{e_i\}$, $\BE \subset \BF_2^n$, of unit vectors $e_i$.
The subset $\BE$ forms a generating system. Each vector $v$ of $\BF_2^n$ is a linear combination of scalars $a_i$ and the $e_i$:
\begin{equation}
v = \sum_{i=1}^n a_i e_i, \enspace a_i \in \BF_2, \enspace e_i \in \BE, \quad \forall v \in \BF_2^n.
\label{def:genSystem}
\end{equation}
Here the addition is modulo 2.

Equation (\ref{def:genSystem}) is used equally for each vector space on any field $K$ using
the operation addition as defined for the field~ $K$ and the $1$ the identity element of the operation multiplication.

Thus the subset $\BE$ spans $\BF_2^n$. In this vector space it is one spanning set,
and the decomposition of a vector $v$ into a linear combination
of unit vectors $e_i$ is unique. 

\subsubsection{Base}
The subset $\BE$ is one base of the vector space $\BF_2^n$. As it is a base of every vector space over a field K.

\subsubsection{Index}
We call $i = 1, \dots, n$ the index of the unit vector $e_i$ in the base.

\subsubsection{Norm via Hamming weight}
The addition in the field $\BF_2^n$ is the addition modulo 2. Hence each summation gives one of the values 0 or 1.
\begin{equation}
\sum_i^n x_i = \left\{ 
\begin{array}{lll}
0, & n \mod 2 = 0, & n \ even, \\
1, & n \mod 2 = 1, & n \ odd. \\
\end{array}
\right.
\end{equation}
So each sum will have either the value 0 or the value 1. From this it follows we cannot use any usual definition of a norm for computing the
length of a vector in $\BF_1^n$.

In coding theory \cite{Hamming1950, wikipedia_en_2025_HammingWeight} the {\em Hamming weight} $w_H$ (number of 1-entries) 
of~ $x \in \{0, 1\}^n$ is defined as:
\begin{equation}
\begin{array}{ll}
w_H : \BF_2^n \rightarrow \BZ, & w_H(x) = | \{i \in \{1, \dots, n\} : x_i = 1\}|,
\end{array}
\end{equation}
and the associated {\em Hamming distance} $d_H$ is:
\begin{equation}
\begin{array}{ll}
d_H : \BF_2^n \times \BF_2^n \rightarrow \BZ, & d_H(x, y) = w_H(x - y).
\end{array}
\end{equation}

If we first apply operation $| \  |$ we map a vector from $\BF_2^n \rightarrow \BR^n$,
by embedding $\BF_2^n = \{0, 1\}^n \subset \BR^n$. On vectors in $\BR^n$ norms are defined and we get:
\begin{equation}
\| x \|_1 = \sum_{i=1}^n | x_i | = w_H(x).
\end{equation}
So the $l_1$-norm on $\BR^n$ see \cite{Forster2016_Analysis_1, Mathworld2025:L1-Norm} equals the Hamming weight on bit vectors.
This norm is sometimes called absolute-value
norm~\cite{wikipedia_en_2025_NormMathematics}.
The value of the norm is an element of the set $\{0, 1, \dots, n\} \subset \BZ \subset \BR$.

In the programming languages C the function is named popcount() and in the language Go it is the function  ~\texttt{OnesCount(uint~x)~uint} in package~\texttt{math/bits}.

Obviously this norm satisfies the axioms of a norm:
\begin{description}
\item[Subadditivity / Triangle inequality] \ \\
$w_H(x+y) \leq w_H(x)+w_H(y) \quad \forall x,y \in \BF_2^n$,
\item[Absolute homogeneity] \ \\
$ w_H(s \cdot x) = s \cdot w_H(x) \quad  \forall s \in \BF_2, x \in \BF_2^n$, 
\item[Positive definiteness] \ \\
$w_H(x) = 0 \Rightarrow x = \0$.
\end{description}
Please note the norm of the unit vectors $\| e_i \| = 1, \forall i \in \{1, \dots n \}$.

\subsubsection{Scalar product} \ \\
We define the scalar product, dot product \cite{Fischer2020, wikipedia_en_2025_DotProduct} or inner product of two vectors as given in equation~(\ref{def:innerproduct}):
\begin{equation}
\begin{array}{ll}
< , > : \BF_2^n \times \BF_2^n \rightarrow \BR, & < x , y > =  \sum_i^n |x_i \cdot y_i| =  \| x \& y \| = w_H,
\end{array}
\label{def:innerproduct}
\end{equation}
The obtained value equals the standard scalar product of $x, y \in \{0, 1\}^n \subset \BR^n$
\begin{equation}
\begin{array}{ll}
< , > : \BR^n \times \BR^n \rightarrow \BR, & < x , y >  = \sum_i^n x_i \cdot y_i,
\end{array}
\end{equation}


We can easily verify by computation the properties of the scalar product:
\begin{description}
\item[Distributivity]
\begin{flalign*}
& <x \oplus x', y > = <x, y> + <x', y>, &\\
& <x, y \oplus y'> = <x, y> + <x, y'>, & \\
& <\lambda x, y > = \lambda <x, y>, & \\
& <x, \lambda y> = \lambda < x, y>, &
\end{flalign*}
\item[Commutativity]
\begin{flalign*}
& <x, y> = < y, x>, &
\end{flalign*}
\item[Positive definiteness]
\begin{flalign*}
& <x, x> \enspace \ge 0, \text{ with } <x,x> = 0 \text{ only if } x = \0, &
\end{flalign*}
\item[Orthogonality]
\begin{flalign*}
& <x, y> = 0, \text{ we say the two vectors are orthogonal.  } &
\end{flalign*}
\end{description} 
In coding theory \cite{Lint1988} the orthogonal bit vector is called the dual code.

\subsubsection{Orthonormal Basis}
With the properties of the norm and the scalar product it follows $\BE$ is an orthonormal base, and this is the only orthonormal base of $\BF_2^n$. 

% \newpage

\printbibliography

\end{document}

