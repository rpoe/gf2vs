\documentclass[a4paper]{article}
\usepackage[utf8]{inputenc}
\usepackage[english]{babel}
\usepackage[T1]{fontenc}
\usepackage{amsmath,amsthm,amssymb}
\usepackage{csquotes}
\usepackage{enumitem}
\usepackage{newtxtext,newtxmath}

% link to database file
% \usepackage{natbib} not working
\usepackage[backend=biber]{biblatex}
\addbibresource{/home/ralf/Projekte/LitDatabase/liblatexbase.bib}

\usepackage[left=1in,right=1in,top=1in,bottom=.5in,includeheadfoot]{geometry}
\usepackage[colorlinks=true,allcolors=black]{hyperref}

\setcounter{biburllcpenalty}{9000}% Kleinbuchstaben
\setcounter{biburlucpenalty}{9000}% Großbuchstaben


\setlength{\parindent}{0pt}
\setlength{\parskip}{2ex}

\renewcommand{\arraystretch}{1.2} % vertical stretch of arrays

\newcommand{\BC}{\mathbb{C}}
\newcommand{\BE}{\mathbb{E}}
\newcommand{\BF}{\mathbb{F}}
\newcommand{\BR}{\mathbb{R}}
\newcommand{\BZ}{\mathbb{Z}}
\newcommand{\0}{\vmathbb{0}}
\newcommand{\1}{\vmathbb{1}}

% for conversion see https://www.glukhov.org/de/post/2025/11/converting-latex-to-markdown-tools-and-workflows/
% pandoc gf2vs.tex -t markdown_strict+tex_math_dollars+raw_tex --bibliography=/home/ralf/Projekte/LitDatabase/liblatexbase.bib --citeproc -o gf2vs.md

\title{Vector space $\BF_2^n$}
\author{Ralf Poeppel \\ \url{mailto:ralf@poeppel-familie.de}}
\date{2026-01-28}

\begin{document}

\maketitle

\begin{abstract}
This article is a supplementary documentation to the Go package \texttt{gf2vs}~\cite{rp:gopkg:gf2vs}. The package implements data types and functions
modeling the vector space $\BF_2^n$.

The vector space $\BF_2^n$ of dimension $n$ is based on the finite field of order $2$, the Galois field $GF(2)$. We use $GF(2)$ to model binary values, or bits, and provide the properties of the vector space of bit vectors.
\end{abstract}

%~\cite{wikipedia_en_2026_GaloisField}

\section{Field $\BF_2$}

\subsection{Supporting Set}

The supporting set of $GF(2) = \BF_2$ is $S = \{0, 1\}$. This set equivalent to $\BZ/\BZ2$ the cyclic set of order 2. This set hold the values of a bit in computer science. In logic we have the boolean values False $F = 0$ and True $T = 1$ \cite{knuth:taocp_v4_f0}.

\subsection{Operations}
The operations of $\BF_2 = \BZ/\BZ2$ are defined modulo~2 see \cite{Fischer2020} ch. 2.2.6.

The operations addition and multiplication of the field $\BF_2$ satisfies the group 
axioms~\cite{wikipedia_en_2026_GroupMathematics}\footnote{We cite Wikipedia for reused wordings.}, both operations are commutative.

\subsubsection{Negation}

\begin{equation}
\begin{array}{r l l l l}
- : \BF_2 \rightarrow \BF_2, & - x = x, & -0 = 0, & -1 = 1. \\
\end{array}
\end{equation}
The negation of $1$ in $\BF_2$ is computed as $(- 1)\mod 2 = 1$

\subsubsection{Value}
We define the mapping value $| x |$ of  an element $x$ of $\BF_2$:
\begin{equation}
\begin{array}{r l l l l}
| x | : \BF_2 \rightarrow \BZ, &| 0 | = 0, & | 1  | = 1. \\
\end{array}
\end{equation}

\subsubsection{Addition}
The addition is named exclusive disjunction in logic and XOR \cite{knuth:taocp_v4_f0, wikipedia_en_2025_ExclusiveOr} in computer science.
The definition of addition is given in equation \ref{def:opaddition} obeying $(1+1) \mod 2 = 0$.
\begin{equation}
\begin{array}{r l l l l}
+ : \BF_2 \times \BF_2 \rightarrow \BF_2, & 0 + 0 = 0, & 0 + 1 = 1, & 1 + 0 = 1, & 1 + 1 = 0, \\
\end{array}
\label{def:opaddition}
\end{equation}

The group axioms \cite{Fischer2020} ch. 2.2.8 for the Group $G = \BF_2$ and the operation addition are satisfied:
\begin{description}
\item[Associativity] \ \\
$\forall a, b, c \in G: (a + b) + c = a + (b + c)$. 
\item[Identity element $e = 0$] \ \\
$\exists e \in G, \forall a \in G: e + a = a \text{ and } a + e = a$, $e = 0$, $e$ is unique.
\item[Inverse element $(-a) = a$] \ \\
$\forall a \in G \enspace \exists (-a) \in G : a + (-a) = e \text{ and } (-a) + a = e$, $e$ identity element, $(-a) = a$ is unique for each $a$.
\item[Commutativity] \ \\
$a + b = b + a$.
\end{description}
So $\BF_2$ with the operation addition is an abelian group.

\subsubsection{Multiplication}
The multiplication is named conjunction in logic and 
AND \cite{knuth:taocp_v4_f0, wikipedia_en_2025_LogicalConjunction} in computer science.
The multiplication is identically defined as in $\BZ$.
\begin{equation}
\begin{array}{r l l l l}
\cdot : \BF_2 \times \BF_2 \rightarrow \BF_2, & 0 \cdot 0 = 0, & 0 \cdot 1 = 0, & 1 \cdot 0 = 0, & 1 \cdot 1 = 1. \\
\end{array}
\label{def:opmultiplication}
\end{equation}

We may use the notation $ab$ instead of $a \cdot b$, omitting the multiplication sign if there is no ambiguity.

The group axioms for the Group $G = \BF_2$ and the operation multiplication are satisfied:
\begin{description}
\item[Associativity] \ \\
$\forall a, b, c \in G: (a \cdot b) \cdot c = a \cdot (b \cdot c)$.
\item[Identity element $e = 1$] \ \\
$\exists e \in G, \forall a \in G: e \cdot a = a \text{ and } a \cdot e = a$, $e= 1$, $e$ is unique.
\item[Inverse element $a^{-1} = a$] \ \\
$\forall a \in G, a \ne 0, \enspace \exists a^{-1} \in G : a \cdot a^{-1} = e \text{ and } a^{-1}b \cdot a = e$, $e$ identity element, $a^{-1} = 1$ is the only inverse element.
\item[Commutativity] \ \\
$a \cdot b = b \cdot a$.
\end{description}
So $\BF_2$ with the operation multiplication is an abelian group.


\subsubsection{Disjunction}

In boolean logic we have the operation disjunction, named OR in computer science \cite{knuth:taocp_v4_f0, wikipedia_en_2025_InclusiveOr}.

\begin{equation}
\begin{array}{r l l l l}
\vee : \BF_2 \times \BF_2 \rightarrow \BF_2, & 0 \vee 0 = 0, & 0 \vee 1 = 1, & 1 \vee 0 = 1, & 1 \vee 1 = 1 \\
\end{array}
\label{def:booleanoperation}
\end{equation}

The operation $\vee$ does not satisfy the group axioms; there is no inverse element. 

\subsection{Field axioms}

The set $K := \BF_2$ with the operations addition and multiplication satisfies the field axioms \cite{Fischer2020} ch. 2.3.3.
\begin{description}
\item[K1] $K$ with the addition $+$ is an abelian group.
\item[K2] $K^* := K\setminus \{0\}$ with the multiplication $\cdot$ for every element of $K^*$ is an abelian group.
\item[K3] distributive property \cite{wikipedia_en_2025_DistributiveProperty} is satisfied $\forall a, b, c \in K$
\begin{equation}
\begin{array}{c}
a \cdot (b + c) = a \cdot b + a \cdot c, \\
(a + b) \cdot c = a \cdot c + b \cdot c. \\
\end{array}
\end{equation}
\end{description}



\section*{Vector space $\BF_2^n$}

We define the vector space
$\BF_2^n$ over the field $\BF_2$ as the set $V$ of vectors $v$ with $n$ elements of the field, together with the
binary operation of vector addition and the binary function of scalar multiplication, see~\eqref{def:vectoroperations}.
This definition is similar to the one in~\cite{wikipedia_en_2025_VectorSpace}.
We use the notation $v := (v_i)$ for the vector $v$ with components $v_i$.

\begin{equation}
u \oplus v = w, \enspace u, v, w \in V, \quad
a \cdot v = w, \enspace a \in \BF_2, \; v, w \in V.
\label{def:vectoroperations}
\end{equation}

We apply the addition element-wise and we multiply the scalar with each element of the vector.

In addition we define two distinguished elements of $\BF_2^n$:

\begin{description}
\item[Zero] \ \\
$\0$ zero vector, all components are $0$.
\item[Ones] \ \\
$\1$ vector, all components are $1$.
\end{description}

The axioms of a vector space are satisfied~\cite{wikipedia_en_2025_VectorSpace}:

\begin{description}
\item[Associativity of vector addition] \ \\
$u \oplus (v \oplus w) = (u \oplus v) \oplus w, \enspace \forall u, v, w \in \BF_2^n$.
\item[Commutativity of vector addition] \ \\
$u \oplus v = v \oplus u, \enspace \forall u, v \in \BF_2^n$.
\item[Identity element of vector addition] \ \\
$\exists \0 \in \BF_2^n : v \oplus \0 = v, \enspace \forall v \in \BF_2^n$.
\item[Inverse elements of vector addition] \ \\
$\forall v \in \BF_2^n \enspace \exists -v \in \BF_2^n : v \oplus (-v) = \0$, and $-v = v$, i.e.\ each vector is its own additive inverse.
\item[Compatibility of scalar multiplication with field multiplication] \ \\
$a (b v) = (ab) v, \enspace a, b \in \BF_2, \; v \in \BF_2^n$.
\item[Identity element of scalar multiplication] \ \\
$1 v = v, \enspace 1 \in \BF_2, \; v \in \BF_2^n$, where $1$ is the multiplicative identity of $\BF_2$.
\item[Distributivity of scalar multiplication with respect to vector addition] \ \\
$a (u \oplus v) = a u \oplus a v, \enspace a \in \BF_2, \; u, v \in \BF_2^n$.
\item[Distributivity of scalar multiplication with respect to field addition] \ \\
$(a \oplus b) v = a v \oplus b v, \enspace a, b \in \BF_2, \; v \in \BF_2^n$.
\end{description}

In this vector space we are not limited to the operations vector addition and scalar multiplication. We can use the Boolean operations as well.

\begin{description}
\item[Complement, Not] \ \\
$\overline v = \1 - v = \1 \oplus v$, swap all bits.
\item[Disjunction, Or] \ \\
$u \vee v = (u_i) \vee (v_i) = (u_i \vee v_i)$, element-wise Or.
\item[Exclusive or, Xor] \ \\
$u \oplus v = u \oplus v = (u_i) \oplus (v_i) = (u_i \oplus v_i)$, element-wise Xor, equal to vector addition.
\item[Conjunction, And] \ \\
$u \wedge v = (u_i) \cdot (v_i) = (u_i \cdot v_i)$, element-wise And.
\end{description}

As we apply the operations element-wise, we satisfy the laws of associativity and commutativity.

We use some more definitions to cover further properties of the vector space:

\begin{description}
\item[Unit vector] \ \\
We define the unit vectors $e_i$, $i = 1, \dots, n$, of the vector space as the vectors where the $i$th element is $x_i = 1$ and all other elements are $0$.
\begin{flalign*}
& e_i = (x_k), & \\
& x_k = \left\{
\begin{array}{ll}
1, & k = i, \\
0, & k \ne i, \\
\end{array}
\right.
\quad x_k \in \BF_2, \enspace e_i \in \BF_2^n. &
\end{flalign*}
Please note the $e_i$ are linearly independent.

\item[Generating system] \ \\
We define the subset $\BE := \{e_i\}$, $\BE \subset \BF_2^n$, of unit vectors $e_i$.
The subset $\BE$ forms a generating system. Each vector $v$ of $\BF_2^n$ is a linear combination of scalars $a_i$ and the $e_i$:
\begin{flalign*}
& v = \sum_{i=1}^n a_i e_i, \enspace a_i \in \BF_2, \enspace e_i \in \BE, \quad \forall v \in \BF_2^n. &
\end{flalign*}
For the summation, we can use either the algebraic addition $+$ or the modulo 2 $\oplus$ addition. 

Thus the subset $\BE$ spans $\BF_2^n$. In this vector space it is one spanning set, and the decomposition of a vector $v$ into a linear combination
of unit vectors $e_i$ is unique. 

\item[Basis] \ \\
The subset $\BE$ is one basis of the vector space $\BF_2^n$.

\item[Index] \ \\
We call $i = 1, \dots, n$ the index of the unit vector $e_i$ in the basis.

\item[Norm] \ \\
We define the norm as a function $p$ of a vector $v \in \BF_2^n$ to be its Hamming weight~\cite{wikipedia_en_2025_HammingWeight}, see \autoref{def:norm}, 
i.e.\ the number of ones in the vector. In some programming languages like C the function is named popcount(). 
\begin{flalign}
 p := \parallel v \parallel :  \BF_2^n \rightarrow \BR, \enspace p = \sum_{i=1}^n v_i
 \label{def:norm}
\end{flalign}
This definition is equivalent to the definition of the $L^1$-norm \cite{Mathworld2025:L1-Norm} of
a vector~$\parallel x \parallel_1$ sometimes called absolute-value
norm~\cite{wikipedia_en_2025_NormMathematics}.
The value of the norm is an element of the set $\{0, 1, \dots, n\} \subset \BR$.
This definition is in accordance with the definition of the norm of the vector space over $\BC$.
Obviously this norm satisfies the axioms of a norm:
\begin{description}
\item[Subadditivity / Triangle inequality] \ \\
$p(x+y) \leq p(x)+p(y) \enspace \forall x,y \in \BF_2^n$,
\item[Absolute homogeneity] \ \\
$ p(s \cdot x) = s \cdot p(x) \enspace  \forall s \in \BF_2, x \in \BF_2^n$, 
\item[Positive definiteness] \ \\
$p(x) = 0 \Rightarrow x = \0$.
\end{description}
Please note the norm of the unit vectors $\parallel e_i \parallel = 1, \forall i \in \{1, \dots n \}$.

\item[Inner product] \ \\
We define the inner product or scalar product of two vectors as given in equation (\ref{def:innerproduct}):
\begin{flalign}
\langle u, v \rangle := 
\left\{ 
\begin{array}{ll}
0, & u \cdot v = \0, \\
1, & else. \\
\end{array}
\right.
\label{def:innerproduct}
\end{flalign}

\item[Orthogonality] \ \\
If $\langle u, v \rangle = 0$, we say the two vectors are orthogonal. Note that the inner product of any vector with $\0$ is $0$.
\end{description}
From this it follows $\BE$ is an orthonormal basis, and this is the only orthonormal basis of $\BF-2^n$.

\newpage

\printbibliography

\end{document}

